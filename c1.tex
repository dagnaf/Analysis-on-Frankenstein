%!TEX root = e.tex
\chapter{Introduction} % (fold)
\label{cha:introduction}
\section{Background and Significance} % (fold)
\label{sec:background_significance}
\begin{text}

\textit{Frankenstein; or the Modern Prometheus}, created by Mary Shelley in 1818, is considered as the first true science fiction novel. Since the early publication, it has raised the awareness of many critics for its profound meanings. However, there are very few reviews on \textit{Frankenstein} from the perspective of moral psychology. But in fact, the novel contains such features as described by the definition of the field. For instance, the moral development. Therefore, based on the theory of moral development, one of the distinguished theories from the perspective, this article explains the moral psychology in the novel to better understand the characters' morality and mentality. Through the analysis, it sheds light on the social bias concerning moral psychology with resonance of the contemporary age. As a result, Shelley's \textit{Frankenstein} and her insight into moral psychology may contribute to the development of today's society.

\end{text}
% section research_background (end)
% \section{Research Significance} % (fold)
% \label{sec:research_significance}
% \begin{text}

%\textbf{TODO:}

% thesis argument | realistic meaning:

% 1 the creature is a victim

% 2 social prejudice, moral development

% \end{text}
% section research_significance (end)
\section{Thesis Structure} % (fold)
\label{sec:thesis_structure}
\begin{text}

This article consists of four chapters. Chapter One gives an concise overview of the article about the background and significance of the research and the structure of the thesis. Chapter Two is about the literature review. It talks about current research status in the field of moral psychology and the others, and introduces the theory used during the analysis coming next. Based on the theory defined in the previous chapter, Chapter Three analyzes the moral psychology in \textit{Frankenstein}. In the beginning, it briefly introduces the life of the author of the novel, Mary Shelley. Then, with a generalized outline of the story, it focuses on the Creature's psychological changes and moral development and draws a conclusion of the analysis in the summary section at the end of the chapter. The last chapter, namely Chapter Four, displays the new findings of the research as well as the limitations and the suggestions for later studies.

\end{text}
% section thesis_structure (end)
% chapter introduction (end)
