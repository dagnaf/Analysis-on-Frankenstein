%!TEX root = e.tex
\chapter{Conclusion} % (fold)
\label{cha:conclusion}
\section{New Findings} % (fold)
\label{sec:new_findings}
\begin{text}

The article explores \textit{Frankenstein; or the Modern Prometheus} from a perspective of moral psychology. According to the theory of Lawrence Kohlberg's Stages of Moral Development, the Creature is identified to be stopped at Stage 2\textonehalf{} among all six stages, a level between the pre-conventional and the conventional. Furthermore, after the analysis of his psychological changes and moral development, it is found that such situation is mostly due to his indulgence in vengeance rather than the society.

Throughout \textit{Frankenstein} by Mary Shelley, she draws attention to the issue of social bias from the perspective of moral psychology. In our present society, there are still prejudices coming in many forms, including appearance, figure, gender, race, age, ethnicity and many features either universally or locally. Of course, more or less, such tacit conventions do undermine our views of the world and moral development, and thus some will challenge but others may obey. Whatever the result is, it is paramount to be emotionally self-controlled in peace. Since it takes time to shape the prejudice as well as to resolve, there is little use in losing control of emotions, which tends to be a species of revenge. Instead, through a tranquil reasoning mind, the rational will construct the world while the emotional assist it, consequently benefiting a society with better moral development.

\end{text}
% section new_findings (end)
\section{Limitations and Suggestions} % (fold)
\label{sec:limitations_and_suggestions}
\begin{text}

In this article, primarily focusing on the Creature's moral psychology, the analysis gives a conclusion on the issue of social bias. However, the issue actually involves two parts: those who are prejudiced and those who prejudices. The former, best represented by the Creature in the novel, has been elaborated in details in Chapter Three, while the latter hasn't. Actually, there are a group of characters in the story belongs to the latter. Although, their bias are scattered segments in Shelley's writing, further dig into these characters moral psychology will probably provide a more fruitful result.
\end{text}
% section limitations_and_suggestions (end)
% chapter conclusion (end)
