%!TEX root = e.tex
\chapter{Analysis of \textit{Frankenstein}} % (fold)
\label{cha:analysis_of_frankenstein}
\section{A Brief Introduction to \textit{Frankenstein}} % (fold)
\label{sec:a_brief_introduction_to_frankenstein}
\subsection{The Author} % (fold)
\label{sub:the_author}
\begin{text}

Mary Shelley (1797--1851), best known for her horror novel \textit{Frankenstein: or, The Modern Prometheus}, was an English novelist in the nineteenth century. Her husband was the romantic poet and philosopher Percy Bysshe Shelley; her father, the political philosopher William Godwin; and her mother, the philosopher and feminist Mary Wollstonecraft. Born in such a family, she received an unusual and advanced education under the large influence of her parent's thoughts, which encourages her political radical and contributes to her literary output later on.

Mary Shelley led a rough life. Her mother died when she was only eleven days old. In 1814, Mary began a romantic relationship with the married Percy Shelley. In spite of the family's objection to their affair, the pair run away together and got married after the suicide of Percy Shelley's first wife. Nevertheless, before the birth of her only surviving child, she had lost three of her children prematurely. Her husband also drowned when sailing during a storm accidentally. In her last decade, she was dogged by illness, probably caused by the brain tumor that was to kill her at the age of 53 \citep{wiki:ms}.

In her whole life, besides the famous \textit{Frankenstein}, other works include the historical novels \textit{Valperga} and \textit{Perkin Warbeck}, the apocalyptic novel \textit{The Last Man}, and her final two novels, \textit{Lodore} and \textit{Falkner}. However, she wrote not only novels but also short stories, dramas, essays, biography and travel books. These works are less known but studied by interested scholars.

\end{text}
% subsection the_author (end)
\subsection{The Work} % (fold)
\label{sub:the_work}
\begin{text}

The story of \textit{Frankenstein} happened in the eighteenth century that Victor Frankenstein, who created a grotesque creature but was horrified and struggled with the destruction till his death. It is written in the form of a frame story \citep{wiki:flkst}. Mainly, there are three narrators in the frames: Captain Robert Walton, Scientist Victor Frankenstein and the Creature.

The novel starts with Captain Walton's introductory frame composed of four letters to his sister about the ambitious expedition to the Arctic, during which he encountered the fainted scientist, looked after him and put down his words as content of the log book.

In the following frames, Frankenstein told the life so far, beginning from his childhood. He was crazy for the origin of life. After pursuit of the elixir formula for many years, he finally reanimated the Creature. But to his disappointment, the appearance was so hideous that he irresponsibly fled away. The Creature had to hide in a low hovel where those who would attack him with stones couldn't see his ugliness. Saddened by rejection, he still believed in love and humanity and tried to make friends with the local villagers by gathering food for them secretly. At first it works. However, when the Creature eventually presented himself in front of his ``friends'', all efforts turned out to be nothing but grief. Hopeless and desperate, he sought revenge on his creator and asked for a female spouse to dwell in the wild away from the habitation of man permanently. Frankenstein agreed reluctantly. While working on the female one , he worried about the premonitions of disaster that their breeding of a race might plague mankind and thus destroyed the work. Wrongly deceived by the promise, the Creature provoked his enemy to chase him up to the North Pole. Frankenstein went after decisively. But he was too weak to resist the coldness and was rescued by Captain Walton.

In closing, Frankenstein died and the Creature appeared. The novel ends with his confession that he would embrace death too, as living remorsefully in a world of unfairness is a torture. In the last two letters, recording what the monster said, Captain Walton resolved to return to his sister.

Interestingly narrated from the first-person perspective and switching among narrators, the book allows its readers to overview characters' point of views, particularly their thoughts and feelings. In the following analysis, it begins from chapter eleven \citepzh{cflksten}{Shelley}, targeting on the Creature's narrative.

\end{text}
% subsection the_work (end)
% section a_brief_introduction_to_frankenstein (end)
\newcommand\citeflkst[2][p]{(\citec{cflksten}{#1. #2})}
\section{Moral Psychology in \textit{Frankenstein}} % (fold)
\label{sec:moral_psychology_in_frankenstein}
\subsection{Psychological Changes} % (fold)
\label{sub:psychological_changes}
\begin{text}

Frankenstein and his creature had the narrative voice in the novel by telling the story of their own history. Articulate as they were, compared to the Creature, the moral psychology of Frankenstein is less depicted since his narration is inclined to spotlight his misfortune and warning of the pursuit of wisdom. In contrast, the Creature revealed his entire psychological changes throughout life, which to some extent even aroused his creator's sense of responsibility.

In the very beginning of his life, the Creature was like any newborns except that he had a gigantic and maturely developed body, and more miserably, he was abandoned. As he recalled with considerable difficulty, ``A strange multiplicity of sensations seized me, and I saw, felt, heard, and smelt at the same time; \dots{} I was a poor, helpless, miserable wretch; I knew, and could distinguish, nothing; but feeling pain invade me on all sides, I sat down and wept.'' \citeflkst[pp]{150--152}. Thus the Creature had not yet internalized what is right or wrong and continued to immerse himself in a world without view of others until his initial contact with mankind. He first entered a small hut. he was surprised by the shriek, but was more enchanted by the shelter and its food \citeflkst[pp]{156--157}. According to Kohlberg, it is blind egocentric with the pre-conventional level. On the one hand, the Creature fell into the Stage 1 since he was punishment driven. After his behavior resulted in bruise by attack of stones, he regretted that ``after my late dearly bought experience, I dared not enter it.'' \citeflkst{157}. On the other hand, the Creature avoided punishment by stealing for his own need of consumption \citeflkst{166}, which expresses the ``what's in it for me'' position in Stage 2.

However, the Creature's view of person concerning intentions was developed when he was watching the De Laceys, a human neighbors, whose gentle manners different from what he had known before struck him chiefly \citeflkst{163}. Therefore, as he discerned how his theft inflicted sufferings on the cottagers, he stopped the practice and apologized to himself with ``I abstained and satisfied myself with berries, nuts and roots,'' \citeflkst{166}. The feeling of sorry did not come from punishment. Instead it was because such kind of action didn't live up to the society's, or the De Laceys' expectations. Hence, the Creature spontaneously stepped across the boarder of the conventional level, specifically Stage 3, with a social relationship perspective as he was longed to join the family. Still, he was afraid of his monstrous face would scare them \citeflkst{165}, which again identifies his moral reasoning at this level.

Hitherto, the Creature almost reached the conventional level, characterized by an acceptance of society's view of right and wrong based on the theory. As for Stage 4, where morality is still predominantly dictated by an outside force, he intended to maintain a functioning society with human by learning its system. However it became ambiguous for his remaining at the fourth stage because at the same time, he had faith in the elimination of the prejudice \citeflkst{172}. Although the belief might be regarded as self-interest to approach the family, he also expressed sympathy to those living in similar conditions and deprecated the vices of mankind. \citeflkst[pp]{179--180, pp. 195--197}. Accordingly, it could be said that he was virtually near the post-conventional level, but sadly he wasn't. Actually, he was far from the top level, or Stage 5, because of the murder of Frankenstein's family, which reflected that his concerns for others was not based on loyalty or intrinsic respect, but rather a ``You scratch my back, and I'll scratch yours.'' mentality, just as what he reasoned with his creator: ``Shall I respect man when he contemns me? \dots{} I will revenge my injuries: if I cannot inspire love, I will cause fear,'' \citeflkst{221}.

But at the end of the story, the Creature cried with sadness that ``still I desired love and fellowship, and I was still spumed. Was there no injustice in this? Am I to be thought the only criminal, when all human kind sinned against me? Why do you not hate Felix, who drove his friend from the door with contumely? Why do you not execrate the rustic who sought to destroy the savior of his child?'' \citeflkst{344}. He obviously realized his immorality, admitting the crime of killing Frankenstein's family and friends. Additionally, he accused those ``virtuous and immaculate beings'' of injustice satirically and still shunned from moral responsibility by blaming it mainly for his self-absorbed psychology, albeit with regrets, as he tried to argue that ``A frightful selfishness hurried me on, while my heart was poisoned with remorse. Think you that the groans of Clerval were music to my ears?'' \citeflkst{341}.

To conclude, the Creature was created in a state of tabula rasa where he had no idea about good and evil. As a clean slate, he learned the hard way to join the society, which was changing him from egoism to altruism, from no view of person to a social perspective. Therefore, there is no denying that the Creature was advancing in his moral development stages. But he was capricious; he accepted the stereotype of the society, but attempted to strive for respects; he committed the murder of the Frankensteins but condemned himself with compunctions. As a result of balancing his emotions, he failed to progress and ultimately regressed.

\end{text}
% subsection psychological_changes (end)
\subsection{Moral Development} % (fold)
\label{sub:moral_development}
\begin{text}

Although it is extremely rare to regress in the six stages of moral development, in Kohlberg's empirical studies of individuals throughout their life, he observed that some had apparently undergone moral stage regression. But this could be resolved either by allowing for moral regression or by extending the theory. Kohlberg chose the latter, postulating the existence of sub-stages in which the emerging stage has not yet been fully integrated into the personality \citep{kohlberg1976moral}. For this reason, in the case of \textit{Frankenstein}, the Creature can be typically noted as Stage 2\textonehalf{}, or Stage 2+, a transition from the pre-conventional level to the conventional level, or Stage 2 to Stage 3 that shares characteristics in both.

Progress through Kohlberg's stages happens as a result of the individual's increasing competence, both psychologically and in balancing conflicting social-value claims, such as fundamental rights, patriotism, respect for human dignity, rationality, sacrifice, individuality, equality, democracy and many others that guide our behavior. Starting from Stage 2, the Creature was getting a social perspective. When he found conflicts not resolved between the social values of reciprocity and the prejudice, he returned to raging envy and vengeance. Thus the social bias has influence upon moral development, especially of those who are similarly at his stage. This is because since even a single stage cannot be skipped \citep{ wiki:kohlberg}, they will probably either progress to Stage 3 or be stuck in Stage 2\textonehalf. However, the Creature's morality is more primarily affected by internal emotions rather than rejections outwards. As the ending of the story has shown, ``then impotent envy and bitter indignation filled me with an insatiable thirst for vengeance. \dots{} but I was the slave, not the master, of an impulse, which I detested, yet could not disobey. \dots{} Evil thenceforth became my good. Urged thus far, I had no choice but to adapt my nature to an element which I had willingly chosen. The completion of my demoniacal design became an insatiable passion.'' \citeflkst{341--342}. And this is why the Creature chose his own sacrifice over the destruction of the world because, to him, it is not the society but himself that is more to blame from the perspective of moral psychology.

Mary Shelley's novel gives rise to the issue of social bias concerning the topics of  moral psychology. While social bias has its impacts on moral development, the self-control of feelings seems to be a dominant factor on the rise and falls of one's sound reasoning. Also Shelley mentioned in the Creature's words that ``The pleasant sunshine, and the pure air of day, restored me to some degree of tranquility; I could not help believing that I had been too hasty in my conclusions. I had certainly acted imprudently. \dots{} I ought to have familiarized the old De Lacey to me, and by degrees to \dots{}'' \citeflkst{209}. So, in fact, the Creature once had chance in retrieving the errors afterwards. By calming down, he could still properly reason at his moral development level in accordance with the stage, which demonstrates that the serene mind does help one's reasoning against conflicts or dilemmas, and thus increasing opportunities of advancement in moral development.

%``Polluted by crimes, and torn by the bitterest remorse, where can I find rest but in death?'' \citeflkst{345}.

%The original driving force behind the Creature's moral development can be listed as the follow three elements. First, learning and knowledge contributes to the Creature's understanding of the world. He learned from the De Laceys, and read partly from , about what is virtue. Second,

%Progress through Kohlberg's stages happens as a result of the individual's increasing competence, both psychologically and in balancing conflicting social-value claims.

%xxx Kohlberg's scale is about how people justify behaviors, and his stages are not a method of ranking how moral someone's behavior is. There should however be a correlation between how someone scores on the scale and how they behave, and the general hypothesis is that moral behavior is more responsible, consistent and predictable from people at higher levels. As is shown before, the Creature's moral responsibility is correspondence of his moral development level.



%In Kolhberg's theory, learning and knowledge contributes to moral development. The Creature's initially progressed as a consequence of touching around the outside world. In Kolhberg's theory, learning and knowledge contributes to moral development. The Creature's initially progressed as a consequence of touching around the outside world.
%skip regress

%moral dilemma

%care respect competence conflicting values

\end{text}
% subsection moral_development (end)
% section moral_psychology_in_frankenstein (end)
\section{Summary} % (fold)
\label{sec:summary_analysis}
\begin{text}

\textit{Frankenstein} is a horror fiction, written by English novelist Mary Shelley. By telling a story about Victor Frankenstein and his creature, Shelley vividly showed her understanding of social bias concerning moral psychology, which can be best traced by looking at the Creature's psychological changes and moral development.

Based on the theory of Lawrence Kohlberg's Stages of Moral Development, the analysis points out the Creature's changes in psychology between the stages, or levels, of moral development. At the early stage, the Creature cared for nothing but himself. But as he progressed through the pre-conventional level, he was confused by social bias and his own principle and eventually regressed. But what is the cause behind his decline in moral development? On the on side, stages cannot be skipped. So he had no way up to higher stages above. On the other side, social bias may claim its contribution partly. However, after the analysis of the Creature's psychology, it is found that the Creature's inclination in giving negative emotions into his reasoning led to his moral failure.

% Throughout \textit{Frankenstein} by Mary Shelley, it draws attention to the issue of social bias from the perspective of moral psychology. In our present society, there are still prejudices coming in many forms, including appearance, figure, gender, race, age, ethnicity and many features either universally or locally. Of course, more or less, such tacit conventions do undermine our views of the world and moral development, and thus some will challenge but others may obey. Whatever the result is, it is paramount to be emotionally self-controlled in peace, since it takes time to shape the prejudice as well as to resolve. Through sound reasoning, the rational constructs the world while the emotional assists it, benefiting a society with better moral development.

\end{text}
% section summary_analysis (end)
% chapter analysis_of_frankenstein (end)
