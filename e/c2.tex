%!TEX root = ../e.tex
\chapter{Literature Review} % (fold)
\label{cha:literature_review}
\section{Previous Studies of \textit{Frankenstein}} % (fold)
\label{sec:previous_studies_of_frankenstein}
\secspacesubsec
\subsection{Moral Psychology Perspective} % (fold)
\label{sub:moral_psychology_perspective}
\begin{text}

Researches on \textit{Frankenstein} from the perspective of moral psychology are not very common, most of which focus on psychological analysis and moral values based on the theories such as moral behavior, moral emotion, moral judgment, moral reasoning and moral responsibility.

One of the recent studies is \textit{Happiness and the Good Life} \citep{martin2012happiness}. This book explores the moral psychology surrounding the Happiness Principle: happiness promotes virtue; unhappiness produces vice. In Chapter Twelve of his book, Felicity in \textit{Frankenstein}, Martin reached a conclusion that the two protagonists did share but actually abused the principle, or the hypothesis, by linking the discussion to the work of positive psychologists. Beginning with the analysis of the text, he explained the moral responsibility for wrongdoing and finally disclosed the fact consistent to Mary Shelley's insights that virtue has its own claim independently of our happiness.

Although reviews are relatively rare from this perspective, some critics related to moral psychology when the topic came to morality and psychology. From the moral perspective, \citetzh{gongwen}{Gong Wen} traced the author's ethical thoughts and claimed that the pursuit of happiness is one of the ways to realize moral ideal. From the psychological perspective, \citetzh{wqymaslow}{Wang Qiuyuan} analyzed the moral behavior based on the psychological theory of Maslow's hierarchy of needs and argued that the Creature wreaked vengeance on the society due to his lack of both love and sense of belonging.

\end{text}
% subsection moral_psychology_perspective (end)
\subsection{Other Perspectives} % (fold)
\label{sub:other_perspectives}
\begin{text}

The contemporary myth, \textit{Frankenstein}, has invited a wide range of interpretations from many other perspectives since its publication. In the field of technology and ecology, many critics approached the novel as a cautionary tale of science. \citet{isaacs1986creation} pointed out the common elements between the myth and the historical development of the atomic bomb, from which he derived the ``Frankenstein scenario'' to better assess how scientific research on recombinant DNA fitted in such model at that time. \citetzh{ecoperspective}{Cao Shanke} put ecocriticism and ethics together. In the article, he inspected the profound thoughts that humans should respect life as well as live harmoniously with nature.

While the popular views are about the consequent upon scientific experimentations \citep{davies2004can}, the story provides for its readers far from this. People also read from the perspective of romanticism, ethics, psychoanalytics and narrative. Recently, they tend to interpret it from the perspective of feminism. \citetzh{femiperspective}{Li Ruochuan and Tan Yuqin} emphasized the status of the female characters to reveal behind the plot their potential power and protests against the patriarchal society.

As more and more thoughts from different perspectives emerge, people start to combine multiple perspectives, or the interdisciplinary perspective, to see whether there exists new findings. These combinations usually include Gothic with feminism, feminism with ecology and ecology with ethics.

\end{text}
% subsection other_perspectives (end)
% section previous_studies_of_frankenstein (end)
\section{Theory of Moral Psychology} % (fold)
\label{sec:theory_of_moral_psychology}
\secspacesubsec
\subsection{Introduction of Moral Psychology} % (fold)
\label{sub:introduction_of_moral_psychology}
\begin{text}

Moral psychology is a field of study in both philosophy and psychology. Some tend to use the term ``moral psychology'' relatively narrowly to refer to the study of moral development \citep{lapsley1996moral}. However, others use the term more broadly to include any topics at the intersection of ethics, psychology, and philosophy of mind. The field mainly covers topics like moral behavior, moral emotion, moral development, altruism and psychological egoism.

Historically, moral psychology began with early philosophers such as Aristotle, Plato and Socrates. They believed that ``to know the good is to do the good''. In their empirical and conceptual researches, they analyzed how people make decisions about issues with regards to moral identity. As the field of psychology separated away from philosophy, the discipline gradually became a formal branch of both fields, expanding to include the role of emotions. Meanwhile, philosophers and psychologists extended the previous empirical measures in researches. Today, structured interviews and surveys such as the Moral Judgment Test (MJT) have been created to study the subject and its development.

Nowadays, it is not a surprise that the study is simultaneously carried forward by philosophers and psychologists. For example, the social psychologist Jonathan Haidt proposed the ``Happiness Hypothesis'', where he expressed opinions on how the contemporary psychology cast light on the moral ideas of the past. For another example, the experimental philosopher Joshua Knobe completed an empirical study recently. The result showed how the way of an ethical problem is phrased dramatically affects an individual's intuitions about the proper moral response to the problem.\footnote{Moral Psychology in \textit{Wikipedia{,} The Free Encyclopedia}. Retrieved from \url{http://en.wikipedia.org/w/index.php?title=Moral_psychology&oldid=646516925}. [Online; accessed 23-February-2015]}

\end{text}
% subsection introduction_of_moral_psychology (end)
\subsection{Definition of Moral Psychology} % (fold)
\label{sub:definition_of_moral_psychology}
\begin{text}

Moral psychology is a novel branch within the field of philosophy and psychology. It investigates people's understanding of morality regarding to their emotions, attitudes and behaviors. One of the well-developed and most important theories of the field is moral development \citepzh{liuyajuan}{Liu Yajuan and Wu Rongxian}. It focuses on individual's construction of morality from a psychological perspective. Furthermore, most researchers currently learn about moral psychology by studying Lawrence Kohlberg, an American psychologist best known for his theory of moral development.

The moral development theory, or Lawrence Kohlberg's Stages of Moral Development, holds that one's morality develops through constructive stages \citep{kohlberg1958development}. These stages are grouped into three levels of two respectively: the pre-conventional, conventional and post-conventional level. The pre-conventional level, especially common in children, judges the morality by direct consequences as described by Stage 1 and Stage 2. People in the former stage are obedience and punishment driven, while those in the latter self-interest driven. However, the conventional level is typical of adolescents and adults. Their senses of morality at Stage 3 are determined by the social consensus or ``golden rule''; Stage 4 the social order or law. Finally, the post-conventional level is marked by one's own principles. Individuals in Stage 5 mutually respect different opinions, rights and values. But in Stage 6, they may take precedent over the society's views because it is right, rather than for their interests or against the penalty. Following the six stages of morality, progressions and regressions happen as a result of psychological changes.

Generally speaking, moral psychology was usually studied by an empirical way while the theory is applied to resolve moral dilemmas. Nevertheless, according to the book, \textit{The Black Guide to Aesthetics}, ``fictions, as well, can be especially useful as sources of insight concerning moral psychology.'' (\citec{kivy2009blackwell}{Kivy, 2009:139}). Kivy did not dig into the text but explicitly pointed out its value reading from the perspective of moral psychology.

\end{text}
% subsection definition_of_moral_psychology (end)
% section theory_of_moral_psychology (end)
\section{Summary} % (fold)
\label{sec:summary_literature}
\begin{text}

Clearly, over the last two centuries, the story of Frankenstein and his creation has been appealing to many new audiences with distinct perceptions. They usually read from the perspective of science, ethics, ecology and psychology. Recently, besides the feminist perspective, some interdisciplinary perspectives also prevail.

However, moral psychology, as a burgeoning interdisciplinary subject, is not so frequently discussed in \textit{Frankenstein} as others. Moral psychology studies both morality and mentality. One of the well-known theory is Kohlberg's six stages of moral development which explains the formation of morality by introducing three moral levels: pre-conventional, conventional and post-conventional level. Its examples often can be found in some real tests concerned about the balance of moral dilemmas. However, the fiction, in fact, also provides useful sources for its development. Thus viewing from the perspective of moral psychology is meaningful.

\end{text}
% section summary_literature (end)
% chapter literature_review (end)

% denial of affection gives rise to a kind of raging envy best understood as a species of revenge. We do not need mountains of data on developmental psychology to appreciate Shelley's perceptiveness, nor an argument. We can discern the inner logic of the Frankenstein Monster's behavior by reflecting on our own feelings.
% Such insight into moral psychology is one of the staples of great fiction. It provides us with knowledge relevant for moral self-understanding and for understanding and judging others morally.

%Conversely. the creature's misery comes not only from his rejection by the villagers but primarily from his giving into the emotions and vices of envy, revenge, hatred
