%!TEX root = ../e.tex
\chapter{Introduction} % (fold)
\label{cha:introduction}
\section{Background and Significance} % (fold)
\label{sec:background_significance}
\begin{text}

\textit{Frankenstein; or, the Modern Prometheus}, written by Mary Shelley in 1818, is considered as the first true science fiction novel. Since the early publication, it has raised the awareness of many critics for its profound meanings. For example, critics usually commented from the perspective of science since the novel involves many scientific knowledge. Ethical perspective are also very popular because it shows both good and evil aspects of a character. However, there are not many reviews on \textit{Frankenstein} from the perspective of moral psychology. But in fact, the novel contains such features as described by the definition of moral psychology. For instance, the moral development. Therefore, based on the theory of moral development, one of the distinguished theories in the field, this article employs the method of textual analysis to explain the moral psychology in the novel. It aims at the interpretations of both morality and mentality. Through analysis of the psychological changes and the moral development, it sheds light on the social bias concerning moral psychology which resonants with the contemporary age. As a result, Shelley's \textit{Frankenstein} and her insight into moral psychology may contribute to the development of the modern society. She suggested in her novel that a serene mind could help cope with social bias, thus preventing stagnation in moral development. Meanwhile, by probing into \textit{Frankenstein} from this perspective, it can help promote the subject of moral psychology.

\end{text}
% section research_background (end)
% \section{Research Significance} % (fold)
% \label{sec:research_significance}
% \begin{text}

%\textbf{TODO:}

% thesis argument | realistic meaning:

% 1 the creature is a victim

% 2 social prejudice, moral development

% \end{text}
% section research_significance (end)
\section{Thesis Structure} % (fold)
\label{sec:thesis_structure}
\begin{text}

This article consists of four chapters. Chapter One gives an concise overview of the article about the background and the significance of the research and the structure of the thesis. Chapter Two is about the literature review. It talks about the current research status in the field of moral psychology and the others, and introduces the theory used in the analysis coming next. Then, based on the theory defined in the previous chapter, Chapter Three analyzes the moral psychology in \textit{Frankenstein}. In the beginning, it briefly introduces the author of the novel, Mary Shelley. After the introduction, with a generalized outline of the story, it focuses on the Creature's psychological changes and moral development, and draws a conclusion in the summary section at the end of the chapter. The last chapter, Chapter Four, displays the new findings of the research as well as the limitations and the suggestions for later studies.

\end{text}
% section thesis_structure (end)
% chapter introduction (end)
