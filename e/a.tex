%!TEX root = ../e.tex

\begin{abstractzh}{《弗兰肯斯坦》,道德心理,科尔伯格}
《弗兰肯斯坦:现代的普罗米修斯》是英国作家玛丽\textperiodcentered{}雪莱的著名小说。主人公维克多\textperiodcentered{}弗兰肯斯坦用科学创造出怪物,但由于丑陋而抛弃了他。怪物诞生后努力融入社会但却遭到孤立,最终走向毁灭。小说自出版后由于其深刻的意义而吸引了许多评论家。本文采用文本分析的定性分析方法,从道德心理学角度,应用科尔伯格道德发展六阶段理论对小说中的道德心理进行分析。文章分析了心理变化和道德发展,最后得出小说中怪物无法控制对于社会偏见的情绪,致使他仅仅停留在道德发展中的2\textonehalf{}阶段,同时也指出以冷静、平和的心态面对社会偏见更有利于道德发展。道德心理学是一门新兴的交叉学科,挖掘《弗兰肯斯坦》其中的道德心理一方面有助于理解复杂的道德发展,推动学科进步,另一方面小说中社会偏见、情感控制对道德的发展有着深远的影响,在当今社会也有启示作用。

\end{abstractzh}

\begin{abstracten}{\textit{Frankenstein}, moral psychology, Kohlberg}
\textit{Frankenstein; or, The Modern Prometheus}, written by Mary Shelley in 1818, is considered as the first true science fiction novel. Since the early publication, it has raised the awareness of many critics for its profound meanings. The novel tells that a grotesque creature was abandoned by his creator Victor Frankenstein because of his unattractive appearance. The Creature wished to be identified by the mankind, but failed and sought revenge. At last he felt regretted and decided to kill himself. The article uses the textual analysis and analyzes the moral psychology based on the theory of Lawrence Kohlberg's Stages of Moral Development to interpret both morality and mentality in \textit{Frankenstein}. Through analysis of the psychological changes and the moral development, it concludes that the Creature's moral development stagnated at Stage 2\textonehalf, which results from his negative emotions towards the social bias. Also, it points out that a serene mind under social bias is beneficial to moral development. As a tentative probe into the burgeoning interdisciplinary research, on the one hand, it may resonates with modern moral development issues on social bias and personal emotions. On the other hand, it can improve the understanding of moral psychology.
\end{abstracten}
