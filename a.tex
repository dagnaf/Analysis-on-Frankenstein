%!TEX root = e.tex

\begin{abstractzh}{《弗兰肯斯坦》,道德心理,科尔伯格}
\vspace{12pt}《弗兰肯斯坦:现代的普罗米修斯》是英国作家玛丽\textperiodcentered{}雪莱的著名小说。主人公维克多\textperiodcentered{}弗兰肯斯坦用科学创造出怪物,但由于丑陋而抛弃怪物,怪物被社会孤立,最终走向毁灭。小说自匿名出版后吸引了许多评论家的眼球。本文从道德心理学角度,应用科尔伯格道德发展六阶段理论分析小说中怪物的心理变化和道德发展。道德心理学是一门新兴的交叉学科,挖掘《弗兰肯斯坦》其中的道德心理一方面有助于理解复杂的道德发展,推动学科进步。另一方面,小说中社会偏见、情感控制对道德发展的有着深远的影响,在当今社会也有启示作用。

\end{abstractzh}

\begin{abstracten}{\textit{Frankenstein}, moral psychology, Kohlberg}
\vspace{12pt}\textit{Frankenstein; or, The Modern Prometheus} is a novel written by English author Mary Shelley about Victor Frankenstein, who created a grotesque creature but was horrified and struggled with the destruction till his death. Critics continues to be attracted since its anonymous publication in the nineteenth century. The article analyzes the moral psychology based on the theory of Kohlberg's Six Stages of Moral Development to better perceive the monster's psychological changes and moral development. As a tentative probe into the burgeoning interdisciplinary research, on the one hand, it may resonates with today's moral development issues on social bias and personal emotions. On the other hand, it can improve the understanding of moral psychology.
\end{abstracten}
